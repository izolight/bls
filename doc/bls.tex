\documentclass[a4paper,12pt]{scrartcl}

\usepackage[utf8]{inputenc}
\usepackage[T1]{fontenc}
\usepackage{url}
\usepackage{fancyhdr}
\usepackage[acronym]{glossaries}
\usepackage{listings}
\usepackage{amsmath}
\usepackage{amssymb}
\usepackage[mathcal]{euscript}

\usepackage[
	backend=biber,
	style=alphabetic,
	sorting=ynt
]{biblatex}
\addbibresource{references.bib}

\fancyhead[R]{}
\fancyhead[C]{Informatikseminar: The BLS Signature Scheme}
\fancyhead[L]{}

\pagestyle{fancy}
\renewcommand{\headrulewidth}{0.5pt}

\fancyfoot[C]{\thepage}
\renewcommand{\footrulewidth}{0.5pt}

\begin{document}

\begin{titlepage}
\begin{center}
\vspace*{3cm}
\vspace{1cm}
\Huge \textbf{Informatikseminar: \\ The BLS Signature Scheme} \\
\vspace{6cm}
\vspace{1cm}
\large Gabor Tanz \\ gabor.tanz@students.bfh.ch \\
\end{center}
\end{titlepage}

\tableofcontents
\pagebreak

\section{Abstract}
\pagebreak

\section{Introduction}
\pagebreak

\section{Prerequisites}
\subsection{Group theory}
A group is a set \(\mathcal{G} = (G, \circ\), inv, \(e\)) with \cite[Slide 2.16]{crypto-slides-haenni} :
\begin{itemize}
	\item a binary operation \(\circ \) that uses two elements of G to produce another element of G:
	\[ G \times G \rightarrow G \]
	\item a unary operation inv:
	\[ G \rightarrow G \]
	\item an identity element:
	\[ e \in G \]
\end{itemize}
The group has to satisfy these properties:
\begin{itemize}
	\item Associativity:
	\[ x \circ (y \circ z) = (x \circ y) \circ z \]
	\[\forall x,y,z \in G \]
	\item Identity:
	\[ e \circ x = x \circ e = x \]
	\[ \forall x \in G \]
	\item Inverse:
	\[ x \circ inv(x) = inv(x) \circ x = e \]
	\[ \forall x \in G \]
\end{itemize}

\subsection{Elliptic curve cryptography (ECC)}
An elliptic curve is a set of points describe by the equation:
\begin{equation*}
y^2 = x^3 + ax + b
\end{equation*}

\subsection{Pairing-based Cryptography}
Pairing-based cryptography is based on pairing functions that map pairs of points on an elliptic curve into a finite field.

\pagebreak

\section{BLS Signatures}
Keygen(): random \[ x\in \mathbb{Z}_{q} \] and set \[ h \leftarrow g_{1}^\alpha\in \mathbb{G}_{1} \] output: \( pk := \)(h) and \( sk := (\alpha) \)


Sign(sk, m): \[ \sigma\ := h^x  h := H(m) \]


Verify(pk, m \(\sigma\)): if \[ e(g_{1},\sigma) = e(pk, H_{0}(m)) \] accept, otherwise reject
\pagebreak

\section{Use cases for BLS Signatures}
\subsection{Secret Sharing}
\subsection{Cryptocurrencies}
\subsection{Blinding}
\pagebreak

\printbibliography

\end{document}
