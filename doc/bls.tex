\documentclass[a4paper,12pt]{scrartcl}

\usepackage[utf8]{inputenc}
\usepackage[T1]{fontenc}
\usepackage{url}
\usepackage{fancyhdr}
\usepackage[acronym]{glossaries}
\usepackage{listings}
\usepackage{amsmath}
\usepackage{amssymb}

\usepackage[mathcal]{euscript}

\usepackage[
	backend=biber,
	style=alphabetic,
	sorting=ynt
]{biblatex}
\addbibresource{references.bib}

\fancyhead[R]{}
\fancyhead[C]{Informatikseminar: The BLS Signature Scheme}
\fancyhead[L]{}

\pagestyle{fancy}
\renewcommand{\headrulewidth}{0.5pt}

\fancyfoot[C]{\thepage}
\renewcommand{\footrulewidth}{0.5pt}

\begin{document}

\begin{titlepage}
\begin{center}
\vspace*{3cm}
\vspace{1cm}
\Huge \textbf{Informatikseminar: \\ The BLS Signature Scheme} \\
\vspace{6cm}
\vspace{1cm}
\large Gabor Tanz \\ gabor.tanz@students.bfh.ch \\
\end{center}
\end{titlepage}

\tableofcontents
\pagebreak

\section{Abstract}
\pagebreak

\section{Introduction}
\pagebreak

\section{Prerequisites}
\subsection{Group theory}
\subsubsection{Definition}
A group is a set \(\mathcal{G} = (G, \circ\), inv, \(e\)) with \cite[Slide 2.16]{crypto-slides-haenni}:
\begin{itemize}
	\item a binary operation \(\circ \), that uses two elements of G to produce another element of G:
	\[ G \times G \rightarrow G \]
	\item a unary operation inv, that inverses \(\circ\):
	\[ G \rightarrow G \]
	\item an identity element, that is neutral:
	\[ e \in G \]
\end{itemize}
The group has to satisfy these properties:
\begin{itemize}
	\item Associativity:
	\[ x \circ (y \circ z) = (x \circ y) \circ z \]
	\[\forall x,y,z \in G \]
	\item Identity:
	\[ e \circ x = x \circ e = x \]
	\[ \forall x \in G \]
	\item Inverse:
	\[ x \circ inv(x) = inv(x) \circ x = e \]
	\[ \forall x \in G \]
	\item This implies Closure:
	\begin{center}
		Any \(x,y \in G \rightarrow x \circ y \) is also in \( G \)
	\end{center}
\end{itemize}

\subsubsection{Examples}
\begin{itemize}
	\item Additive Groups
	\begin{itemize}
		\item Integers: \( (\mathbb{Z},+,-,0) \)
		\item Real numbers: \( (\mathbb{R},+,-,0) \)
		\item Elliptic curve over a prime field: \( (E_{a,b}(\mathbb{Z}_{p}),+,-, \mathcal{O}) \)
	\end{itemize}
	\item Multiplicative Groups
	\begin{itemize}
		\item Integers modulo \( n: ( \mathbb{Z}_{n}^*, \times_{n}, ^{-1}, 1) \)
		\item Non-zero rational numbers: \( ( \mathbb{Q} \setminus 0, \times, ^{-1}, 1) \)
	\end{itemize}
\end{itemize}

\subsubsection{Cyclical Groups}

\subsection{Elliptic curve cryptography (ECC)}
An elliptic curve is a set of points \((x,y)\) satisfying the equation\cite[Slide 6.4]{crypto-slides-haenni}:
\begin{equation*}
y^2 = x^3 + ax + b
\end{equation*}

\subsection{Pairing-based Cryptography}
Pairing-based cryptography is based on pairing functions that map pairs of points on an elliptic curve into a finite field.\cite{security-wiki-pbc}
\newline
With two values (P and Q) a pairing function is defined as \( e(P,Q) \)

\pagebreak

\section{BLS Signatures}
\subsection{Definition}
The BLS Signature Method (named after Boneh, Lynn and Shacham) uses bi-linear pairing with an elliptic curve group.
\newline
For a pairing function to be bi-linear, it has to satisfy these properties:
\begin{itemize}
	\item commutativity for additive and multiplicative operations
	\item distributivity for multiplicative operations
\end{itemize}

\subsection{Properties}

\subsection{Signature Generation and validation}
We have the following functions
\begin{itemize}
	\item Keygen()
	\begin{center}
		random \( x\in \mathbb{Z}_{q} \) and set \( h \leftarrow g_{1}^\alpha\in \mathbb{G}_{1} \) output: \( pk := \)(h) and \( sk := (\alpha) \)
	\end{center}
	\item Sign \( (sk, m) \)
	\[ \sigma\ := h^x  h := H(m) \]
	\item Verify\( (pk, m \sigma) \)
	\begin{center}
		if \( e(g_{1},\sigma) = e(pk, H_{0}(m)) \) accept, otherwise reject
	\end{center}
\end{itemize}

\pagebreak

\section{Use cases for BLS Signatures}
\subsection{Secret Sharing}
Shamir secret sharing https://blog.dash.org/secret-sharing-and-threshold-signatures-with-bls-954d1587b5f
\subsection{Cryptocurrencies}
\subsubsection{Advantages}
\begin{itemize}
	\item short Signatures
	\item aggregation of signatures and keys
\end{itemize}

\subsubsection{Uses}
\begin{itemize}
	\item Dashpay
\end{itemize}

\subsection{Blinding}
Given signature \( \langle \sigma, g^x \rangle \) on message \( h \), we can blind the signature and public key \( g^x \):
\[ e(\sigma^b,g) = e(h,g)*{xb} = e(h,g^{xb}) \]
Thus \( \sigma^b \) is a valid signature for the derived public key \( (g^x)^b \) with blinding value \( b \in \mathbb{Z}_{q} \).\cite[PKI Slide 12]{crypto-slides-grothoff}
\pagebreak

\printbibliography

\end{document}
